\documentclass{article}
\title{Blackjack}
\usepackage[utf8]{inputenc}
\usepackage{hyperref} % for hyperlinks
\usepackage{amssymb}
\usepackage{authblk}
\usepackage{minted}   % for code linting
\usepackage{amsmath}   % big brackets
\usepackage{hyperref}
\author{Joshua Ortiga \\
\and
Xin Li \\
\and
Jonathan Le}
\begin{document}
\maketitle
\section{Background}
\label{sec: Background}

	\subsection{Blackjack Rules}
	\label{sec: Blackjack Rules}

		The rules of Blackjack are simple. Every card in Blackjack (excluding Jokers) are weighed to be worth a certain
		amount of points. All cards from 2-9 are valued at their respective ranks. Face cards (Jack, Queen, and King) 
		are valued at 10 points. Finally, Aces are valued at either 1 or 11 points, and can switch their values at
		any stage of the game. 

		The game starts with the dealer and the player drawing two cards, with the dealer only showing one of their cards.
		In Blackjack, you have the ability to either hit or stand. Hitting will draw you another card, while standing will 
		end your turn, revealing the other dealer's card, and letting the dealer hit until they are at 17 points or higher.

		The goal of the game is to get 21, or closer to 21 points than the dealer. If you have more points than the dealer,
		and your points do not exceed 21, or if the dealer's points exceed 21, you win. If the points are equal, you draw, 
		and finally if you have less points than the dealer, and the dealer's points do not exceed 21, you lose.

		One final aspect of Blackjack to note is that in a real game, money and betting is involved. However, this research paper
		will not cover that aspect of the game. Therefore, surrendering, doubling down and splitting bets will not be considered
		factors for the scope of this paper.

	\subsection{Hypothetical Rule Changes}
	\label{sec: Hypothetical Rule Changes}

		Since Blackjack has existed for a long time now, it is a solved game. However, the details of this paper will discuss a hypothetical rule
		change to the game. What if the game was played such that no face cards (Jacks, Queens, and Kings) were not in the decks? Would
		the META (most effective tactic available) of the game change? Or would the optimal strategy remain the same?

\section{Results}
\label{sec: Results}

        \subsection{Optimal Strategy for Vanilla Blackjack}
	\label{Optimal Strategy for Vanilla Blackjack}

        Resource - \href{https://www.kjartan.co.uk/games/pix/cards/Blackjack%20full%20guide.pdf}{Blackjack Strategy Guide}

        \begin{itemize}
            \item hand\_hands = player has no aces in hand
            \item soft\_hands = player has ace(s) in hand
            \item Key = dealer's face up card
            \item Value pairs = total values in player's hand, range to hit
        \end{itemize}

        \begin{minted}{python}
                hard_hands = {
                    1: (4,17), # Dealer has ace
                    2: (4,12),
                    3: (4,12),
                    4: (4,11),
                    5: (4,11),
                    6: (4,11),
                    7: (4,16),
                    8: (4,16),
                    9: (4,16),
                    10: (4,16),
                    11: (4,17), # Dealer has ace
                }
        \end{minted}

        \begin{minted}{python}
                soft_hands = {
                    1: (12,18), # Dealer has ace
                    2: (12,18), 
                    3: (12,18), 
                    4: (12,18), 
                    5: (12,18), 
                    6: (12,18), 
                    7: (12,17), 
                    8: (12,17), 
                    9: (12,18), 
                    10: (12,18), 
                    11: (12,18), # Dealer has ace
                }
        \end{minted}

        \subsection{Modified Strategy For No-Face Blackjack}
	\label{Modified Strategy For No-Face Blackjack}

        Due to the lack of face cards in the deck, the modified optimal strategy increases the player's hit range significantly.
        
        \vspace{0.25cm} 
        
        \hspace{-0.55cm}When the dealer's face up card is 2, 3, 4, 5 or 6, and the player has a hard hand, the player should hit on 16, and stand on 17.

        \vspace{0.25cm} 

        \hspace{-0.55cm}When the dealer's face up card is an ace and the player has a hard hand, the player should hit on 17, and stand on 18.

        \vspace{0.5cm}
        
        \hspace{-0.55cm}Optimal Modified Hard Hands:
        \begin{minted}{python}
                hard_hands = {
                        1: (4,17), # Dealer has ace
                        2: (4,16),
                        3: (4,16),
                        4: (4,16),
                        5: (4,16),
                        6: (4,16),
                        7: (4,16),
                        8: (4,16),
                        9: (4,16),
                        10: (4,16),
                        11: (4,17), # Dealer has ace
                    }
        \end{minted}

        \vspace{0.25cm}

        \hspace{-0.55cm}The strategy for soft hands does not change for the modified blackjack game. 

        \subsection{Developing the Modified Strategy}
	\label{Developing the Modified Strategy}

        By removing all the face cards in the deck, the deck will consist of 12 less 10-values cards. 

        \vspace{0.25cm}
        
        \hspace{-0.5cm}Number of each card in the modified deck:
        \begin{itemize}
            \item Ace: 4
            \item Two: 4
            \item Three: 4
            \item Four: 4
            \item Five: 4
            \item Six: 4
            \item Seven: 4
            \item Eight: 4
            \item Nine: 4
            \item Ten: 4
        \end{itemize}

        \vspace{0.25cm}

        \hspace{-0.5cm}Total number of cards in modified deck: 40.

        \hspace{-0.5cm}Test results after running a million test cases with the modified no-face deck:

        \begin{minted}{json}
                {
                    "lose": 509432,
                    "draw": 81430,
                    "win": 409138,
                    "dealer_bust": {
                        "10": 12127,
                        "2": 20734,
                        "3": 22150,
                        "4": 24376,
                        "5": 26588,
                        "6": 26146,
                        "7": 16392,
                        "8": 14458,
                        "9": 13691,
                        "Ace": 3518,
                        "total": 180180
                    },
                    "dealer_card_freqs": {
                        "10": 100209,
                        "2": 100328,
                        "3": 100232,
                        "4": 99618,
                        "5": 100033,
                        "6": 99812,
                        "7": 100654,
                        "8": 99443,
                        "9": 99926,
                        "Ace": 99745,
                        "total": 1000000
                    }
                }
        \end{minted}

        ---Insert the graphs here---

        \vspace{0.25cm}

        \hspace{-0.5cm}\textbf{Reasoning behind modified hard hands strategy when dealer's face up card is an ace (assuming player total hand is consisting of only two cards):}

        \vspace{0.25cm}

        \hspace{-0.5cm}In the two cases where the Dealer has an ace, and player doesn't, the chances of the the dealer's face down card being an ace is $\frac{3}{37} = 0.0810810810810811$. Combinations of cards in players hand that add up to 17: (7, 10), (8, 9).

        \vspace{0.25cm}
        
        \hspace{-0.5cm}Number of cards remaining in deck within the bust threshold (assuming dealer's other card is a card that's 4 or below): $(6 \cdot 4) - 2 = 22$. Number of cards remaining in the deck below the bust threshold: $(4 \cdot 4) - 2 = 14$. If the player stands at 18, the dealer has to bust for the player to win.
        
        \vspace{0.25cm}
        
        \hspace{-0.5cm}Probability of player busting: $\frac{22}{36}$ = 0.6111

        \vspace{0.25cm}

        \hspace{-0.5cm}Probability of dealer busting calculated with $\frac{dealer\_bust}{dealer\_card\_freqs}$ Key being starting face up card:
        \begin{itemize}
            \item Ace: $\frac{3518}{99745} = 0.035$
        \end{itemize}

        \vspace{0.5cm}

        \hspace{-0.5cm}\textbf{Reasoning behind modified hard hands strategy when dealer's face up card is not an ace (assuming player total hand is consisting of only two cards):}

        \vspace{0.25cm}

        \hspace{-0.5cm}In the rest of the cases where the Dealer doesn't have an ace, and the player doesn't, the chances of the dealer's face down card being an ace is $\frac{4}{37} = 0.1081081081081081$. Combinations of cards in players hand that add up to 16: (6, 10), (7, 9), (8, 8).

        \vspace{0.25cm}
        
        \hspace{-0.6cm} Number of cards remaining in deck within the bust threshold (assuming dealer's other card is a card that's 4 or below): $(5 \cdot 4) - 2 = 18$. Number of cards remaining in deck below the bust threshold: $(5 \cdot 4) - 2 = 18$. If the player stands at 17, the dealer has to bust for the player to win.
        
        \vspace{0.25cm}
        
        \hspace{-0.5cm}Probability of player busting: $\frac{18}{36}$ = 0.5

        \vspace{0.25cm}

        \hspace{-0.5cm}Probability of dealer busting calculated with $\frac{dealer\_bust}{dealer\_card\_freqs}$ Key being starting face up card:
        \begin{itemize}
            \item Two: $\frac{20734}{100328} = 0.21$
            \item Three: $\frac{22150}{100232} = 0.22$
            \item Four: $\frac{24376}{99618} = 0.24$
            \item Five: $\frac{26588}{100033} = 0.27$
            \item Six: $\frac{26146}{99812} = 0.26$
            \item Seven: $\frac{16392}{100654} = 0.16$
            \item Eight: $\frac{14458}{99443} = 0.15$
            \item Nine: $\frac{13691}{99926} = 0.14$
            \item Ten: $\frac{12127}{100209} = 0.12$
        \end{itemize}

        \hspace{-0.5cm}Although the dealer's bust rates is much lower than the player's, if the player were to hit at 18 and above, the chances of busting would be much higher and not ideal. The mean bust rate of the dealer without an ace as a starting card would be: $\frac{sum\_of\_all\_bust\_rates}{9} = 0.197$ 
        This is equivalent to the player having a 20\% chance of winning the game and an even higher chance of getting a tie. 

        \vspace{0.25cm}

        \hspace{-0.5cm}Now if we observe the dealer's bust rates, we can see that the bust rates peak when they have a 5 as a starting face up card. Additionally, the bust rates are much lower from 7-10 (starting card) as compared to 2-6 for the dealer. This is because the dealer has a much higher chance of drawing a card that puts them at 17 or above when the dealer already has a starting face-up card that's more than 6, which lowers their chances of busting since they will not draw as much. The chances of the dealer busting if his starting card is 6 or below is much higher since there is a good chance that they could draw more times and not be at 17 yet. 

        \vspace{0.25cm}

        \hspace{-0.5cm}When the dealer's starting face-up card is an ace, this means it could either be an 11 if it doesn't put them over 21, or a 1 if it does. This explains why the bust rate for when the dealer has an ace is so low at 3.5\%. The amount of times they draw is dependent on their current ace's value, which lowers their bust rate significantly.
        

\section{Summary}
\label{sec: Summary}

	Lorem Ipsum	

\section{Works Cited}
\label{sec: Works Cited}

\href{https://github.com/hosua/blackjack-cs241}{Blackjack Simulator}

\href{https://en.wikipedia.org/wiki/Glossary_of_game_theory}{Glossary of Game Theory}

\href{https://www.kjartan.co.uk/games/blackjack.htm}{Optimal Strategy Guide}

\href{https://www.kjartan.co.uk/games/pix/cards/Blackjack%20full%20guide.pdf}{Optimal Strategy Chart}

\end{document}