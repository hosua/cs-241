\documentclass{article}
\title{HW 5}
\usepackage[utf8]{inputenc}
\usepackage{hyperref} % for hyperlinks
\usepackage{amssymb}
\usepackage{authblk}
\usepackage{minted}   % for code linting
\usepackage{amsmath}   % big brackets
\author{Joshua Ortiga \\
\and
Xin Li \\
\and
Jonathan Le}
\begin{document}

\maketitle
\textbf{Problem 5.1.} Find a function $f(n)$ that will provide the $n^{\text{th}}$ Fibonacci 

number.
\begin{center}
	$f(n) = f(n-1)+f(n-2)$
\end{center}

\vspace{0.5cm}

\textbf{Problem 5.2.} Take the following functions and order from smallest to 

largest with respect to their $\Theta$ function.

\vspace{0.25cm}

\textit{Functions:}

log$(x)$, $\sqrt[3]{x}$, $\sqrt[500]{x}$, $1$, $x^2$log$(x)$, $x^{10^{10^{20}}}$, $x^{x}$,
$\pi(x)$, $x^{2}$, $1.1^{x}$, $x$, $x$log$(x)$, $1.9^{x}$, 

$2.1^{x}$, $x!$

\vspace{0.25cm}

\textit{Sorted:}

$1$, $\sqrt[500]{x}$, $log$(x)$, \sqrt[3]{x}$, $\pi(x)$, $x$, $x$log$(x)$, $x^2$, $x^2$log$(x)$, $x^{10^{10^{20}}}$, $1.1^{x}$, $1.9^{x}$, $2.1^{x}$, 

$x!$, $x^{x}$

\vspace{0.25cm}

$1.1^{x}$, $1.9^{x}$, $2.1^{x}$ \textit{are relatively equal with each other.}

\vspace{0.25cm}

$\sqrt[500]{x}$, $\sqrt[3]{x}$ \textit{are relatively equal with each other.}

\vspace{0.5cm}

\textbf{Problem 5.3.} Assume that the merge operation requires $\Theta(n)$ work, what 

is the $\mathcal{O}$ of merge-sort? (Prove it using the method we showed in class)

\vspace{0.25cm}

$T(1) = 1$

\vspace{0.25cm}

\textit{When splitting the array into a left and right half:}

\vspace{0.1cm}

$\Rightarrow$ $T(n) = 2T(\frac{n}{2}) + cn$

$\Rightarrow$ $T(n) = 2T(\frac{n}{2}) + cn$

$\Rightarrow$ $2T(\frac{n}{4}) + c(\frac{n}{2})$

\vspace{0.2cm}
 
\textit{After performing the merge sort on the left and right half, and then merging}

\textit{the two sorted halves of the array together:}

\vspace{0.1cm}

$\Rightarrow$ $T(n) = 2 [2T(\frac{n}{4}) + c(\frac{n}{2})] + cn$

$\Rightarrow$ $4T(\frac{n}{4}) + 2cn$

\vspace{0.25cm}

\textit{General form:}

\vspace{0.1cm}

$T(n) = 2^k \cdot T(\frac{n}{2^k}) + kcn$ where $n=2^k$, $(\frac{n}{2^k}) = 1$, $k = $log$_2(n)$

\vspace{0.25cm}

\textit{Then substitute k into the general form of T(n) function:}

\vspace{0.1cm}

$T(n) = n \cdot T(1) + $log$_2(n) \cdot cn$

$T(n) = \mathcal{O}(n \cdot $log$(n))$ \hspace{1cm} $\blacksquare$

\vspace{0.5cm}

\textbf{Problem 5.4.} LeetCode problem: Given two input linked lists that 

represent numbers in reverse order (where the first element of the list is the 

last digit, etc), write a function that adds the two numbers together and 

returns a new linked list containing their sum.

\begin{minted}{python}
	# Definition for singly-linked list.
	# class ListNode:
	#   def __init__(self, x):
	#     self.val = x
	#     self.next = None
	class Solution:
		def addTwoNumbers(self, l1: ListNode, l2: ListNode) -> ListNode:
			x1 = 0
			x2 = 0
			count = 0

			while l1!=None and l2!=None:
				x1 += l1.val * 10**count
				x2 += l2.val * 10**count
				l1 = l1.next
				l2 = l2.next
				count += 1

			while l1!=None:
				x1 += l1.val * 10**count
				l1 = l1.next
				count += 1

			while l2!=None:
				x2 += l2.val * 10**count
				l2 = l2.next
				count += 1

			ans = x1 + x2
			head = ListNode(ans % 10)
			ans //= 10
			n1 = head
			
			while ans:
				n1.next = ListNode(ans % 10)
				ans //= 10
				n1 = n1.next

			return head
\end{minted}

Give a function (or two) that represents the number of operations (where 

operations are defined as how we listed them in class) as a function of the 

lengths $m$,$n$ of our two input lists $l_1$, $l_2$ respectively. You might need to 

use two functions to represent upper/lower bounds. Then provide the most 

simplified class of $\Omega$, $\mathcal{O}$, and $\Theta$ that this code belongs to.

\vspace{0.5cm} 
\textit{Best case scenario:}

\textit{Let n = len($l_1$) = len($l_2$).}

\vspace{0.25cm} 

\textit{For all operations outside of any loops, the sum of operations = 14.}

\vspace{0.25cm} 

\textit{Number of operations in the first while loop = 15$\cdot$n.}

\vspace{0.25cm} 

\textit{Both loops are never executed since all nodes are visited in the first loop}

\textit{when m = n. Therefore, number of operations for the second and third }

\textit{loops = 2 (For each loop's comparison operation).}

\vspace{0.25cm} 

\textit{Number of operations in the final while loop = 9$\cdot$(n-1).}

\vspace{0.25cm} 

\textit{$\therefore$ function for best case is: }

\begin{center}
	\textit{14+15n+2+9(n-1)}
\end{center}

\vspace{0.5cm} 

\textit{Worst case scenario:}

\textit{Let m = len($l_1$), n = len($l_2$).}

\vspace{0.25cm} 

\textit{Let x = a number comprised of the values in $l_1$.}

\textit{Let y = a number comprised of the values in $l_2$.}

\vspace{0.25cm} 

\textit{In the worst case, two numbers x and y will be two numbers selected, s.t.}

\textit{the result of the adding x+y yields a result s.t. the number of digits of the}

\textit{result = max(m,n)+1. For example, 1+9=10 or 99+1=100.}

\vspace{0.25cm} 

\textit{For all operations outside of any loops, the sum of operations = 14.}

\vspace{0.25cm} 

\textit{Number of operations in the first while loop = 15$\cdot$min(m,n).}

\vspace{0.25cm} 

\textit{The while loop iterating over the linked list with less nodes is skipped, so}

\textit{only one operation is executed for the comparison operation.}

\vspace{0.25cm} 

\textit{The while loop for the linked list containing more nodes is not skipped, and}

\textit{it executes 8$\cdot$abs(m-n) operations.}

\vspace{0.25cm} 

\textit{The number of operations in the final while loop = 9$\cdot$max(n,m) operations.}

\vspace{0.25cm} 

\textit{$\therefore$ the function to find the number of operations in the worst case is:}

\begin{center}
	\textit{14+(15$\cdot$min(m,n))+1+(8$\cdot$abs(m-n))+(9$\cdot$max(n,m))} 
\end{center}

$\blacksquare$

\vspace{0.5cm} 

\textbf{Problem 5.5.} \textit{Extra credit}: This problem is a bit challenging so it will 

be worth 10 points of extra credit. For large values of $n$ the exact number 

of operations $f(n)$ approaches the following function 

\begin{center}
	$\lim\limits_{n \to +\infty} f(n) = k \cdot n!$
\end{center}

where $n!$ is the factorial of $n$. Find the value of $k$. Because it's so much 

extra credit, there will be no partial credit for this.

\begin{minted}{python}
	def crappyFactorial(n):
		if n == 0:
			return 1
		elif n == 1:
			return 1
		total = 0
		for (i=0; i < n; i = i + 1):
			total = total + crappyFactorial(n-1)
		return total
\end{minted}

\vspace{0.5cm}

\begin{center}
	\textit{Left as an exercise for the grader.}
\end{center}

\end{document}
