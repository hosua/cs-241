\documentclass{article}
\title{HW 3}
\usepackage[utf8]{inputenc}
\usepackage{hyperref} % for hyperlinks
\usepackage{amssymb}
\usepackage{authblk}
\usepackage{minted}   % for code linting
\usepackage{amsmath}   % big brackets
\author{Joshua Ortiga \\
		\and
		Xin Li \\
		\and
		Jonathan Le}
\begin{document}

\maketitle
\textbf{Problem 3.1.} Let $f(x)=tan(x)$, and define $g(x)$ as the following: 

\begin{center}
$
	\hfill
	\begin{cases}
	 0 \;\;\;\; x \leq 0 \\ 
	 1 \;\;\;\; x > 0
	\end{cases}
	\hfill (1)
$
\end{center}

on which domain A and codomain B $g(x)$ an inverse function for $f(x)$?

\vspace{0.5cm}

Proof by contradiction

\vspace{0.25cm}

Let $f(x) = tan(x)$ and $x = 45$.

\vspace{0.25cm}

Assume $g(x) = f(x)^{-1}$.

\vspace{0.25cm}

Let $y = f(x) = f(45) = 1$.

\vspace{0.25cm}

$g(y) = g(1) = 1$.

\vspace{0.25cm}

Since $g(y) \neq f(x)$, $g(x)$ is not $f(x)^{-1}$ $\blacksquare$

\vspace{0.5cm}

\textbf{Problem 3.2.} Take a clock and replace all the numbers on the face with 

numbers $1 \leq n \leq 19$, such that if you sum all the numbers up you get

30. For example, a valid clock face would be [4, 3, 6, 3, 1, 1, 1, 2, 1, 3, 3, 

2] starting with the first number in the 12 position, and going around 

clockwise from there. Prove that for any valid sequence on the clock, 

that there must exist a consecutive sequence that adds to exactly 10.

\vspace{0.25cm}

Assume that there is a wrapping list of integers from 1-19 (inclusive) that

add up to 30, without any sequence that add up to 10. Since the lowest 

sequence that doesn't add up to 10 is = $[1,3,4,1,3,4,1,3,4,1,3,4]$ and the 

sum of the following list of integers is 32 which greater than 30. 

Since lowering any integer in this list results in at least 1 sequence that adds 

to 10, there is no list of integers that can add up to 30 that won't result in 

sequence of 10. $\blacksquare$

\vspace{0.5cm}

\textbf{Problem 3.3.} Let $a, b, \in \mathbb{Z}_+$ and $gcd(a,b) = 1$. Prove that the decimal 

expansion of $\frac{a}{b}$ must either terminate or will loop infinitely.

\vspace{0.25cm}

Assume there are 2 numbers $a, b$ (positive integers $gcd(a,b) = 1$) $s.t.$

$a$ and $b$ do not terminate or loop infinitely. $\pi$ does not terminate or 

loop infinitely, and there are no 2 numbers that divide into $\pi$.

$\therefore$ there are no 2 numbers that will not terminate and not loop infinitely 

because there are no 2 numbers that divide into $\pi$. $\blacksquare$


\vspace{0.5cm}

\textbf{Problem 3.4.} Let $\mathbb{Z}/n$ be the quotient set of $\mathbb{Z}$ under the relation 

$x \equiv y(mod\;n)$ (a.k.a a complete system of incongruent residues). 

Suppose that:

\begin{center}
	\hfill $\mathbb{Z}/n = \{[x_1],[x_2],...,[x_n]\}.$ \hfill (2)
\end{center}

If $gcd(k,n) = 1$, prove that:

\begin{center}
	\hfill $\mathbb{Z}/n = \{[x_1],[x_2],...,[x_n]\} = \{[kx_1],[kx_2],...,[kx_n]\}.$ \hfill (3)
\end{center}

\vspace{0.25cm}

False. 

\vspace{0.25cm}

Proof by contradiction

\vspace{0.25cm}

Since $\mathbb{Z}/n$ is a complete system of incongruent residues, the remainders in 

the set has to all be different from each other.

\vspace{0.25cm}

Let $n = 4,\;\;\;\;z = ...8,9,10,11...$

\vspace{0.25cm} 

\begin{center}
	$\mathbb{Z}/n = \{0,1,2,3\}$
\end{center}

Let $k = 3$.

\vspace{0.25cm} 

\begin{center}

	$gcd(k,n) = gcd(3,4) = 1$
	
	\vspace{0.25cm} 

	$\{k(0),k(1),k(2),k(3)\} = \{0,3,6,9\} $

	\vspace{0.25cm} 

	$\{0,1,2,3\} \neq \{0,3,6,9\}$

\end{center}

There is a contradiction, $\therefore$ it is false. $\blacksquare$


\end{document}
